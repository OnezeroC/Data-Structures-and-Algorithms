\documentclass{article}
\usepackage{geometry}
\usepackage{xeCJK}
\geometry{a4paper, margin=1in}

\title{List类功能测试报告}
\author{陈奕林}
\date{\today}

\begin{document}

\maketitle

\section{简介}
本文档报告了对 \texttt{List} 类的功能测试结果。测试的目的是验证链表类中常见操作的正确性,包括插入、删除、遍历以及迭代器的功能。

\section{测试的功能}
在 \texttt{List.cpp} 文件中测试了以下 \texttt{List} 类的功能:

\begin{itemize}
    \item \textbf{构造和初始化:}
        \begin{itemize}
            \item 使用 \texttt{List<int> myList\{10, 20, 30, 40, 50\}} 初始化包含多个元素的链表。
        \end{itemize}
    \item \textbf{压入与弹出操作:}
        \begin{itemize}
            \item \texttt{push\_back(60)} 在链表末尾添加元素。
            \item \texttt{push\_front(5)} 在链表头部添加元素。
            \item \texttt{pop\_back()} 删除链表最后一个元素。
            \item \texttt{pop\_front()} 删除链表第一个元素。
        \end{itemize}
    \item \textbf{访问操作:}
        \begin{itemize}
            \item \texttt{front()} 返回链表第一个元素。
            \item \texttt{back()} 返回链表最后一个元素。
        \end{itemize}
    \item \textbf{插入与删除:}
        \begin{itemize}
            \item \texttt{insert(iterator, value)} 在迭代器位置插入元素。
            \item \texttt{erase(iterator)} 删除迭代器位置的元素。
        \end{itemize}
    \item \textbf{大小与空检查:}
        \begin{itemize}
            \item \texttt{size()} 返回链表中的元素数量。
            \item \texttt{empty()} 检查链表是否为空。
        \end{itemize}
    \item \textbf{移动操作:}
        \begin{itemize}
            \item 使用 \texttt{std::move()} 测试移动构造函数。
            \item \texttt{clear()} 用于清空链表的所有元素。
        \end{itemize}
\end{itemize}

\section{总结}
通过对 \texttt{List} 类的测试,验证了其正确处理元素的插入、删除和通过迭代器访问的能力。此外,该类的内存管理功能(如移动构造函数、析构函数等)表现符合预期。

\end{document}
