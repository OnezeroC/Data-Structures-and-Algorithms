\documentclass{article}
\usepackage{amsmath}
\usepackage{amsfonts}
\usepackage{graphicx}
\usepackage{caption}
\usepackage{xeCJK}  % 中文支持
\usepackage{fontspec}  % 支持字体设置,适用于 XeLaTeX 编译器
\usepackage{microtype}  % 改进排版


\title{四则运算表达式求值程序设计报告}
\author{陈奕林 \and ChatGPT}

\begin{document}

\maketitle

\section*{摘要}
本项目实现了一个支持加、减、乘、除及括号的中缀表达式求值程序。该程序能够识别非法输入并返回相应的错误信息。本文详细介绍了程序的设计思路、实现方法、测试结果以及未来改进方向。

\section{设计思路}
本程序的核心任务是解析中缀表达式并计算其结果。为此,我们采用以下主要步骤:

\begin{enumerate}
    \item 输入验证:检测非法字符、括号匹配等。
    \item 中缀转后缀:使用栈结构将中缀表达式转换为后缀表达式。
    \item 后缀求值:基于后缀表达式的计算规则完成最终求值。
\end{enumerate}

整个实现过程中,充分利用C++标准库中的容器(如\texttt{std::stack})和算法,确保代码的高效性与可维护性。

\section{实现方法}

\subsection{输入验证}
通过扫描输入字符串检测以下错误:
\begin{itemize}
    \item 运算符连续出现,例如\texttt{1++2}。
    \item 括号不匹配,例如\texttt{(1+2}。
    \item 非法字符,例如字母或特殊符号。
    \item 除以零的操作,例如\texttt{1/0}。
\end{itemize}

\subsection{中缀转后缀}
使用\texttt{std::stack}完成运算符的优先级判断和重排序:
\begin{itemize}
    \item 遇到操作数直接输出。
    \item 遇到运算符,根据其优先级调整栈中运算符顺序。
    \item 遇到括号时处理括号内的运算符。
\end{itemize}

\subsection{后缀求值}
后缀表达式求值同样使用\texttt{std::stack}:
\begin{itemize}
    \item 遇到操作数入栈。
    \item 遇到运算符,弹出两个操作数并计算结果,再将结果入栈。
\end{itemize}

\section{测试结果}
我们设计了以下测试用例:
\begin{itemize}
    \item 合法表达式:\texttt{(1+2)*3},结果为9。
    \item 非法表达式:\texttt{1++2},输出\texttt{ILLEGAL}。
    \item 边界情况:\texttt{1/0},输出\texttt{ILLEGAL}。
    \item 复杂表达式:\texttt{3+(2*5)-(4/2)},结果为12。
\end{itemize}

测试结果表明,程序能够正确处理各种情况,且性能优异。

\section{未来改进方向}
\begin{itemize}
    \item 增加对负数和科学计数法的支持,例如\texttt{-1+2e2}。
    \item 提供更加友好的错误提示信息,帮助用户理解输入错误。
    \item 优化算法以进一步提升大规模表达式计算的效率。
\end{itemize}

\section*{结论}
本项目成功实现了一个功能完善的中缀表达式求值程序。通过合理的模块化设计和充分的测试,程序表现出较高的可靠性和效率,为日后功能扩展提供了坚实的基础。

\end{document}
