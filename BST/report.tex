\documentclass{article}
\usepackage{amsmath}
\usepackage{amsfonts}
\usepackage{graphicx}
\usepackage{fontenc}
\usepackage{xeCJK}

\title{AVL树的\texttt{remove}函数实现报告}
\author{陈奕林 \and ChatGPT}
\date{\today}

\begin{document}

\maketitle

\section*{AVL树的\texttt{remove}函数实现报告}

\texttt{remove}函数的设计旨在删除节点的同时保持AVL树的平衡特性。AVL树的特性确保每个节点的左、右子树高度差(即平衡因子)不超过1,从而使插入、删除和查找操作的时间复杂度均能保持在 $O(\log n)$ 级别。为此,在删除节点后,AVL树需要通过适当的旋转来恢复平衡。

\subsection*{主要实现步骤}

\begin{enumerate}
    \item \textbf{删除节点}:
    \begin{itemize}
        \item \texttt{remove}函数使用递归方式查找并删除指定的节点。
        \item 若删除的节点有两个子节点,则使用右子树的最小节点(即后继节点)替代被删除节点的值,以保持二叉搜索树的性质。
        \item 删除节点完成后,返回到父节点,逐层检查并更新高度。
    \end{itemize}
    
    \item \textbf{调整平衡}:
    \begin{itemize}
        \item 在每次递归返回时调用\texttt{balance}函数,判断是否需要旋转以恢复平衡。
        \item 如果左子树比右子树高2,则执行右旋\texttt{rotateWithLeftChild}或双旋转\texttt{doubleWithLeftChild}。
        \item 如果右子树比左子树高2,则执行左旋\texttt{rotateWithRightChild}或双旋转\texttt{doubleWithRightChild}。
    \end{itemize}

    \item \textbf{高度更新}:
    \begin{itemize}
        \item 每次平衡操作后,重新计算旋转后节点的高度。节点高度为其左右子节点高度中的较大值加1。
    \end{itemize}
\end{enumerate}

\subsection*{平衡操作的旋转逻辑}

\begin{itemize}
    \item \textbf{单旋转}:用于修正简单的左右不平衡。例如,当左子树比右子树高2时,执行单右旋,使得树重新平衡。
    \item \textbf{双旋转}:用于修正复杂的不平衡情况。当左子树的右子树较高(或右子树的左子树较高)时,需先对不平衡子树进行子旋转,再对祖父节点进行旋转,从而实现平衡。
\end{itemize}

通过上述步骤,\texttt{remove}函数在删除节点后能有效恢复AVL树的平衡,使树的高度保持在对数级增长,从而确保操作的高效性。

\end{document}
