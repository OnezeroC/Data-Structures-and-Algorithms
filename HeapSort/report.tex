\documentclass{article}
\usepackage{amsmath}
\usepackage{amsfonts}
\usepackage{graphicx}
\usepackage{caption}
\usepackage{xeCJK}  % 中文支持
\usepackage{fontspec}  % 支持字体设置,适用于 XeLaTeX 编译器
\usepackage{microtype}  % 改进排版

\setCJKmainfont{SimSun}  % 设置中文字体,可以根据需要调整

\title{HeapSort 实现与性能对比报告}
\author{陈奕林 \and ChatGPT}

\begin{document}
	\maketitle
	
	\section{整体设计思路}
	本报告介绍了一个自定义的堆排序实现,并通过对比 `std::sort\_heap()`来分析性能差异。堆排序是一种基于完全二叉树的排序算法,它通过将待排序元素构建成一个堆,然后反复将堆顶元素与最后一个元素交换并调整堆,达到排序目的。
	
	在实现中,我们首先定义了一个 `HeapSort` 类,该类模板化设计,能够对任意类型的容器(如 `std::vector`)进行堆排序。实现了基本的堆操作,包括 `heapify`、`buildHeap` 和 `sort` 方法。
	
	\section{核心算法}
	堆排序的基本流程如下:
	\begin{enumerate}
		\item 构建最大堆:将待排序数组转换为最大堆。
		\item 交换堆顶元素与数组末尾元素,堆的大小减一。
		\item 调整堆结构,使其重新成为最大堆。
		\item 重复上述过程,直到堆的大小为1。
	\end{enumerate}
	
	\section{测试流程}
	测试分为以下几步:
	
	\begin{itemize}
		\item 生成四种不同的测试序列:
		\begin{enumerate}
			\item 随机序列:随机生成的 1000000 个整数。
			\item 有序序列:从 0 到 999999 的递增序列。
			\item 逆序序列:从 999999 到 0 的递减序列。
			\item 部分重复序列:每 1000 个元素重复一次的序列。
		\end{enumerate}
		\item 使用 HeapSort 对每个序列进行排序,并记录排序所花费的时间。
		\item 使用 std::sort\_heap() 对每个序列进行排序,并记录排序所花费的时间。
		\item 输出每次排序的时间,并检查排序结果是否正确。
	\end{itemize}
	
	\section{测试结果}
	以下是不同序列的排序时间(单位:秒),并与 std::sort\_heap() 的时间进行了对比:
	
	\begin{table}[h!]
		\centering
		\begin{tabular}{|c|c|c|}
			\hline
			\textbf{序列类型} & \textbf{堆排序时间} & \textbf{std::sortheap 时间} \\
			\hline
			随机序列 & 0.0847808秒 & 0.0323649秒 \\
			有序序列 & 0.0437664 & 0.030287秒 \\
			逆序序列 & 0.0445614秒 & 0.0288435秒 \\
			部分重复序列 & 0.0525432秒 & 0.0294509秒 \\
			\hline
		\end{tabular}
		\caption{堆排序与std::sort\_heap的性能对比}
	\end{table}
	
	\section{时间复杂度分析}
	堆排序的时间复杂度为 $O(n \log n)$,其中 $n$ 是待排序序列的大小。堆排序的关键操作是构建堆和对堆进行调整(heapify)。构建堆的时间复杂度为 $O(n)$,而每次调整堆的时间复杂度为 $O(\log n)$,因此总的时间复杂度为 $O(n \log n)$。
	
	与 std::sort\_heap() 的效率对比:
	
	std::sort\_heap() 的实现通常是通过堆排序或者其他优化方法实现的,其时间复杂度也是 $O(n \log n)$。
	在本次测试中,HeapSort 和 std::sort\_heap() 在不同类型的数据上表现出不同的性能差异。这些差异可能与实际堆的实现、内存访问模式以及排序过程中内部优化策略有关。std::sort\_heap() 通常在优化上会比手动实现的堆排序更加高效,尤其是在某些数据结构优化和缓存优化方面。
	
	\section{总结}
	本报告实现了一个自定义的堆排序算法,并与 std::sort\_heap() 进行了性能对比。从测试结果来看,在处理不同类型的序列时,std::sort\_heap() 通常表现得更加高效。这主要是由于 std::sort\_heap() 经过高度优化,能够在实际应用中提供更好的性能。然而,手动实现堆排序为理解和实现堆排序提供了深入的思路,并可以在需要时进行灵活调整。
	
\end{document}
